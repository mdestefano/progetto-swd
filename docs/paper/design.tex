% !TeX spellcheck = en_US
\section{Empirical Study Design}\label{sec:design}
The \textit{goal} of the study is to perform an historical analysis of the test-suites related to components affected by \asmells in open-source systems, with the \textit{purpose} of assessing whether the quality of these test suites decreases when \asmells are introduced.
Moreover, the study aims to asses how the fault proneness of the considered components varies when these smells occur.
The \textit{perspective} is for both academics and practitioners: while the former ..., the latter are interested in maintaining certain code components. 

\subsection{Context}
The context of our study is made up of \asmells and software systems.
Among the currently known \asmells, we decided to put our focus on the following: [LISTA DI SMELLS].
We chose these because they all occur at class level, so we could conduct our study at the same level of granularity.
Moreover, \cyclic and \hublike are well-known smells and object of a great number of studies[CITARE KELLY].
However, for the opposite reason, we chose to focus on [RESTANTI SMELLS], since, as explained by [CITARE KELLY], they have never been studied.